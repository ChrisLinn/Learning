\documentclass[a4paper,10pt]{article}
%\documentclass[a4paper,10pt]{scrartcl}

\usepackage[utf8]{inputenc}
\usepackage{listings}

\title{Save-the-words Haskell}
\author{Chris Lin}
\date{\today}

\begin{document}
\maketitle

\begin{enumerate}
 \item The easiest way: Haskell platform
 \begin{enumerate}
  \item GHC: the most widely used Haskell compiler.\\
  How to use:
  \begin{itemize}
   \item Start:
   \\\texttt{ghci}
   \\\texttt{:set prompt "ghci> "}
   \item Load up a file (provided \textbf{myfunctions.hs}): 
   \\\texttt{:l myfunctions}
   \item Reload:
   \\\texttt{:r}
  \end{itemize}
 \end{enumerate}
 \item Basic knowledge:
 \begin{enumerate}
  \item Always surround a negative number with parentheses.
  \item inequality symbol:
  \\\texttt{/=}
  \\(watch out for the difference between \texttt{4} and \texttt{"4"})
  \item functions are called by writing the function name, a space and then the parameters, separated by spaces. For examples,
  \\\texttt{min 9 10}
  \\So,
  \\\texttt{bar (bar 3)} means \texttt{bar(bar(3))} in C.
  \\And there is no \texttt{bar(bar 3)} in Haskell.
  \item Function application has the highest precedence, which means these two statements are equivalent:
  \\\texttt{succ 9 + max 5 4 + 1}
  \\\texttt{(succ 9) + (max 5 4) + 1}
  \item If a function takes two parameters, we can also call it as an infix function by surrounding it with backticks.
  \\\texttt{div 92 10}
  \\\texttt{92 \`{}div\`{} 10}
  \item 
 \end{enumerate}
\end{enumerate}
\end{document}
