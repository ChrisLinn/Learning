\documentclass[a4paper,10pt]{article}
%\documentclass[a4paper,10pt]{scrartcl}

\usepackage[utf8]{inputenc}
%\usepackage{listings}

\title{Save-the-word Haskell}
\author{Chris Lin}
\date{\today}

\begin{document}
\maketitle

\begin{enumerate}
 \item The easiest way: Haskell platform
 \begin{enumerate}
  \item GHC: the most widely used Haskell compiler.\\
  How to use:
  \begin{itemize}
   \item Start (use them both, one by one):
   \\\texttt{ghci}
   \\\texttt{:set prompt "ghci> "}
   \item Load up a file (provided \textbf{myfunctions.hs}): 
   \\\texttt{:l myfunctions}
   \item Reload:
   \\\texttt{:r}
  \end{itemize}
 \end{enumerate}
 \item Basic knowledge
 \begin{enumerate}
  \item Always surround a negative number with parentheses.
  \item inequality symbol:
  \\\texttt{/=}
  \\(watch out for the difference between \texttt{4} and \texttt{"4"})
  \item function
  \begin{itemize}
  \item Functions are called by writing the function name, a space and then the parameters, separated by spaces. For examples,
  \\\texttt{min 9 10}
  \\So,
  \\\texttt{bar (bar 3)} means \texttt{bar(bar(3))} in C.
  \\And there is no \texttt{bar(bar 3)} in Haskell.
  \item Function application has the highest precedence, which means these two statements are equivalent:
  \\\texttt{succ 9 + max 5 4 + 1}
  \\\texttt{(succ 9) + (max 5 4) + 1}
  \item If a function takes two parameters, we can also call it as an infix function by surrounding it with backticks.
  \\\texttt{div 92 10}
  \\\texttt{92 \`{}div\`{} 10}
  \item Write your own functions:
  \begin{itemize}
   \item how to make functions (contents in \textbf{myfilename}.hs):
   \\\texttt{doubleMe x = x + x}
   \\\item how to make use of it:
   \\\texttt{:l }\textbf{myfilename}
   \\\texttt{doubleMe 9}
   \\\item some examples:
   \\\texttt{doubleMe x = x + x}
   \\\texttt{doubleUs x y = x*2 + y*2}
   \\\\(by having these, we can also run: 
   \\\texttt{doubleUs 28 88 + doubleMe 123})
   \\\\(we can also redefine the function \texttt{doubleUs} as:
   \\\texttt{doubleUs x y = doubleMe x + doubleMe y})
   \\\\\item Functions in Haskell don't have to be in any particular order, so it doesn't matter if you define \texttt{doubleMe} first and then \texttt{doubleUs} or if you do it the other way around.
   \\\item \texttt{if} statement:
   \\\texttt{doubleSmallNumber x = if x > 100}
   \\\texttt{......................then x}
   \\\texttt{......................else x*2}
   \\\\(Each '.' indicates a space. Because I fail to create spaces :P)\\
   \begin{itemize}
    \item the \texttt{else} part is mandatory in Haskell.
    \item \texttt{if} statement in Haskel is an expression:
    \\\texttt{doubleSmallNumber' x = (if x > 100 then x else x*2) + 1}
    \\\textbf{notes:} That apostrophe (') doesn't have any special meaning. It's ok in a function name. We usually use ' to either denote a strict version of a function (one that isn't lazy) or a slightly modified version of a function or a
variable.\\
   \end{itemize}
   \item what is more:
   \begin{itemize}
    \item Functions can't begin with uppercase letters.
    \item When a function doesn't take any parameters, we usually say it's a definition (or a name):
    \\\texttt{conanO'Brien = "It's a-me, Conan O'Brien!"}
   \end{itemize}
  \end{itemize}
  \end{itemize}
  \item lists
  \begin{itemize}
   \item elements need to be of the same type
   \item make a list:
   \\\texttt{let lostNumbers = [4,8,15,16,23,42]}
   \\(Doing let a = 1 inside GHCI is the equivalent of writing a = 1 in a script and then loading it.)
   \item  strings are lists of characters
   \item putting two lists together:
   \begin{itemize}
    \item \texttt{[1,2,3,4] ++ [9,10,11,12]}
    \\(take a while if the left one is too big)
    \\\item \texttt{'A':" SMALL CAT"}
    \\(instantaneous)
    \\\item \texttt{}{:} takes a number and a list of numbers or a character and a list of characters, whereas \texttt{++} takes two lists. 
    \\\\(So if you're adding an element to the end of a list with ++ , you have to surround it with square brackets so it becomes a list.)
    \\\\(\texttt{[1,2,3]} means \texttt{1:2:3:[]} or \texttt{1:2:[3]})\\
   \end{itemize}
   \item \texttt{[]} , \texttt{[[]]} and \texttt{[[] , [] , []]} are different:
   \\an empty list
   \\a list that contains one empty list
   \\a list that contains three empty lists
  \end{itemize}

  \end{enumerate}
\end{enumerate}
\end{document}
